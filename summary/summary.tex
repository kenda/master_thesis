\chapter{Summary}
\label{cha:summary}

This thesis presented an approach for answering several drug related questions based on Linked Open Data knowledge bases.
Section \ref{sec:objectives} stated the objective to provide an easy to use interface.
This objective could be achived what is proofed by the evaluation projects.
Therefore the developers or later the respective users have nothing else to know than the appropriate drug names or drug codes.
All the required knowledge about the structure of the different knowledge bases is abstracted by SEDRI.
How easy the usage of SEDRI is showed the two evaluation projects.
They both showed also that SEDRI is usable independently of the programming language or other implementation details of the projects.
This is due to the integration and component characteristics of a web-based HTTP interface.
Thereby the two main objectives of section \ref{sec:objectives} are achieved.

The problems that were worked out in section \ref{sec:problem} could be avoided mostly by the infrastructure of SEDRI.
The first problem was insuffient integration which is mostly fixed by the Linked Data approach.
Having that said the integration could be further improved by implementing Entity Matching mechanisms or the like.
By using Linked Data as a knowledge foundation also the second problem -- redundancy -- is mostly fixed.
This is due to the several domain specific knowledge bases where only a small part of the data is redundant.
The last problem of possibly incorrect data is rather not applicable to Linked Open Data.
The reason is that most of the biomedical knowledge bases are converted from already existing data bases, eg. Drugbank.
Therefore the correctness of the knowledge depends on the original data sets.

On the downside especially the evaluation by Dispedia showed some points that are missing to reach a state where the information provided by SEDRI are ready for the everyday professional use.
Belonging to these points is the fact that most of the information in the knowledge bases are only available in English.
Another point is the granularity of the provided data, for example the missing severity of drug-drug interactions as carved out in section \ref{sec:impr-data-level}.
By regarding the additional downtimes of some knowledge bases every now and then -- which might be fixed at application level -- the question whether Linked Open Data is an appropriate knowledge foundation for drug management has currently rather to be answered negatively. 
But it has to be mentioned that this applies to the professional use with the current state of the data.
Private persons might already have a benefit by getting the information that their medication contains drug-drug interactions independently of any severity.
Also the language problem mentioned aboved does not apply for completely English applications.
Therefore in this case the current state of the data might be sufficient.

These points and the other outlined possible improvements of chapter \ref{cha:discussion} show that the domain of drug management in combination with semantic data contains enough potential for further work and theses in the next years.

%  \begin{itemize}
% % \item professioneller einsatz fraglich
% % \item lod evtl noch nicht geeignet
% % \item genug potential für weiterentwicklung
%  \item auf weitere probleme aus einleitung eingehen ob gelößt oder nicht\todo{}
%  \item insufficient integration
%  \item redundancy
%  \item correctness of knowledge
%  \end{itemize}

%%% Local Variables: 
%%% mode: latex
%%% TeX-master: "../thesis"
%%% End: 
