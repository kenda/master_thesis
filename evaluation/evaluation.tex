
\chapter{Evaluation}
\label{cha:evaluation}

\section{Dispedia}
\label{sec:dispedia}
\begin{itemize}
\item Anforderungen für gute patientensicht (erfassung über barcode etc)?
\item Anwendungsfälle und Vorgehen beschreiben
\end{itemize}
\section{DrugMan}
\label{sec:portal}

\subsection{Subject}
\label{sec-1}

\begin{itemize}
\item schwer für ärzte überblick über verfügbare medikamente am markt zu behalten
\item oft bilden sich aus erfahrung paare von medikamenten zu gegebener krankheit die nur selten dann erneuert werden (Quelle oä?)
\item 
\item ärzte verschreiben oft unabhängig voneinander medikamente
\item patient hat möglichkeit rezeptfreie medikamente zusätzlich einzunehmen
\item bpz verweisen nur grob auf mögliche interaktionen
\begin{itemize}
\item oft nur gruppen von medikamenten genant, nicht medikamente konkret
\end{itemize}
\end{itemize}
\subsection{Problem}
\label{sec-2}

\begin{itemize}
\item patienten bekommen arzneimittel verschrieben auf grund der erfahrung der ärzte nicht evidenzbasiert bzw. an aktuellen studien ausgerichtet
\item mehr oder minder blindes vertrauen des patienten den ärzten gegenüber
\item 
\item bei n arzneimittel gibt es (n$^2$-n)/2 mögliche interaktionen
\begin{itemize}
\item 2->1, 3->3, 4->6, 5->10
\end{itemize}
\end{itemize}
\subsection{Motivation}
\label{sec-3}

\begin{itemize}
\item transparente möglichkeit der unterstützung von drug management eines patienten
\item bereicherung der persönlichen freiheit durch nachvollziehbarkeit und kontrolle der eigenen arzneimitteltherapie
\end{itemize}

%%% Local Variables: 
%%% mode: latex
%%% TeX-master: "../thesis"
%%% End: 
