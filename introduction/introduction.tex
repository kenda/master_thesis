\chapter{Introduction}
\label{cha:introduction-1}

\section{Subject}
\label{sec-1}

\begin{itemize}
\item verschiedene anwendungen im  bereich arzneimittelmngtm
\item überschneidende funktionalität
\item 
\item keine endanwendungen auf lodd basierend bekannt
\end{itemize}
\section{Problem}
\label{sec-2}

\begin{itemize}
\item unterschiedliche implementierungen
\item unters. quellen
\item ausgesagt werden soll aber das selbe
\item 
\item umfangreiches schemawissen nötig um anfragen an lod stellen zu können
\end{itemize}
\section{Motivation}
\label{sec-3}

\begin{itemize}
\item durch einheitliche schnittstelle soll datenqualität vereinheitlicht werden
\item einstiegsbarriere für neue anwendunge senken (cpoe)
\item 
\item vereinfachen der möglichkeit wissen aus lodd zu nutzen
\item (praxiseinsatz von lodd)
\end{itemize}
\section{Objectives}
\label{sec-4}

\begin{itemize}
\item bereitstellen einer API zur unterstützung des drug management life cycles basierend auf lod(d)
\item evaluation anhand verschiedener anwendungsfälle
\begin{itemize}
\item dispedia
\item eigenes portal zur arzneimittel-verwaltung
\end{itemize}
\end{itemize}

%%% Local Variables: 
%%% mode: latex
%%% TeX-master: "../thesis"
%%% End: 
