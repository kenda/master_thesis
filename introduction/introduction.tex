\chapter{Introduction}
\label{cha:introduction-1}

\section{Subject}
\label{sec:subject}

The prescription and application of drugs in health care is one of the most important parts in the healing process of  a patient.
If this process is not managed accurately consequences from financial losses to  human harms are possible.
In Germany estimations state that annually 25.000 patients die because of medication errors \cite{pharzeit10} (status 2010).
In consequence it is essential that -- as a base to avoid these problems -- all available knowledge about drugs and their characteristics is freely accessible and usable.
This process of publishing medical -- drug related -- data has already started in the last years with projects like XXX or XXX.
%That is important because so it is possible for everyone to build third-party applications on top of this data.
Although it is possible to build third-party applications on top of this data, such applications are very rare at this time.
In the case of LODD only small projects exist like XXX or Pharmer \cite{khalili2013pharmer}.

Besides these third-party projects mainly clinical applications implement specific drug related functionalities.
For example a \textit{Patient Information System} stores all the details about the prescibed drugs of a certain patient and may want to support the nurses by offering informations about the route of application or the maximum dosage.
Another application, like a \textit{Computerized Physician Order Entry System} that selects and prescribes drugs, want to check interactions of the possible drugs.
And for both of the applications the side effects of a certain drug may be of interest.
These examples show that many different medical applications implement drug related functionalities.
And many of them also share the same requirements because there is a set of common questions, like \textit{``What are the side effects of drug X?''}.
But ignoring these shared requirements, many of the applications provide their own -- mostly proprietary -- knowledge bases.

% \begin{itemize}
% \item verschiedene anwendungen im  bereich arzneimittelmngtm
% \item überschneidende funktionalität
% \item 
% \item keine endanwendungen auf lodd basierend bekannt
% \end{itemize}

\section{Problem}
\label{sec:problem}

In section \ref{sec:subject} the diversity of applications that implement drug related functionalities was mentioned.
This goes along with the fact that multiple knowledge bases are used which actually should provide the same data. 
This leads to several problems.

The first problem -- insufficient integration -- follows from the the different scopes and the different expressivity of the knowledge bases.
This means there are many knowledge bases for special domains like side effects or alternative medicine.
But if one knowledge base wants to use the data of another special domain this is usually not possible.
Therefore it has to express the data on their own and thats the starting point of the next problem: redundancy.
Many drug related knowledge bases include a set of common facts that are essential for this domain.
The third problem is the correctness of the knowledge.
While the number of persons that validate the data of a proprietary knowledge source is limited, projects like Wikipedia has shown the power of an open review process in the last years.

A common problem that third-party applications using these data sources have is the extensive schema knowledge that is required to use the data efficiently.
% \begin{itemize}
% \item unterschiedliche implementierungen
% \item unters. quellen
% \item ausgesagt werden soll aber das selbe
% \item 
% \item umfangreiches schemawissen nötig um anfragen an lod stellen zu können
% \end{itemize}

\section{Motivation}
\label{sec:motivation}

If the given problems of section \ref{sec:problem} could be solved this would lead to a better data quality in the first place.
Data quality contains in this context the expressivity, completeness and correctness of the knowledge.

Another consequence would be the reduced starting barrier for application developers.
They don't have to know the whole schema of the knowledge base, they have just to know how to use the given interface.
This would hopefully lead to many interesting applications on top of the given data.

% \begin{itemize}
% \item durch einheitliche schnittstelle soll datenqualität vereinheitlicht werden
% \item einstiegsbarriere für neue anwendunge senken (cpoe)
% \item 
% \item vereinfachen der möglichkeit wissen aus lodd zu nutzen
% \item (praxiseinsatz von lodd)
% \end{itemize}

\section{Objectives}
\label{sec:objectives}

Emerged from the motivation this thesis shall provide an web-based \textit{Application Programming Interface} (API).
This API will support the process of answering drug related questions according to the WHO drug management life cycle.

This thesis shall also provide two use cases where the usage of the API will be demonstrated and evaluated.
The first use case is the integration in Dispedia -- an information system in the complex field of rare diseases \cite{elze2013dispedia}.
The second use case is a personal drug management portal built around this API to support privat persons managing their prescribed drugs.

% \begin{itemize}
% \item bereit\-stellen einer API zur unter\-stützung des drug manage\-ment cycles basierend auf lod(d)
% \item evaluation anhand verschiedener anwendungsfälle
% \begin{itemize}
% \item dispedia
% \item eigenes portal zur arzneimittel-verwaltung
% \end{itemize}
% \end{itemize}

%%% Local Variables: 
%%% mode: latex
%%% TeX-master: "../thesis"
%%% End: 
