\chapter{Introduction}
\label{cha:introduction-1}

\section{Subject}
\label{sec:subject}

The prescription and application of drugs in health care is one of the most important parts in the healing process of  a patient.
If this process is not managed accurately consequences from financial losses to  human harms are possible.
In Germany estimations state that annually 25.000 patients die because of medication errors \cite{pharzeit10} (status 2010).
Managing drugs means to support the different phases drugs passes during the healing process.
The \textit{World Health Organization} (WHO) published a Drug Management Cycle which defines four different phases \cite{who2004}.
Details about this Drug Management Cycle are given in section \ref{sec:drug-management}. 
An important base to implement a proper drug management and therefore to avoid the problem of medication errors is that all available knowledge about drugs and their characteristics is freely accessible and usable.
This process of publishing medical -- drug related -- data has already started in the last years with projects like \textit{Linking Open Drug Data} (LODD) \cite{jentzsch2009linking} or BioPortal \cite{whetzel2011bioportal}.
LODD is a project that links the knowledge of several vocabularies by recognizing common entities and freely provides this linked data.
More details about this approach are given in section \ref{sec:linked-open-data}.
BioPortal offers a platform where people or projects serve their biomedical ontologies and knowledge bases.
These knowledge bases can in turn be mapped so that the same entities in different vocabularies are identified.
%That is important because so it is possible for everyone to build third-party applications on top of this data.
Although it is possible to build third-party applications on top of this data, such applications are very rare at this time.
In the case of LODD only small projects exist like DiseaseCard \cite{oliveira2004diseasecard} or Pharmer \cite{khalili2013pharmer}.

Besides these third-party projects mainly clinical applications implement specific drug related functionalities.
For example a \textit{Patient Information System} stores all the details about the prescribed drugs of a certain patient.
Therefore a nurse could be supported in the process of drug application by offering information about the route of application or the maximum dosage.
Another application, like a \textit{Computerized Physician Order Entry System} that selects and prescribes drugs, could check interactions of the possible drugs.
And for both of the applications the side effects of a certain drug may be of interest.
These examples show that many different medical applications implement drug related functionalities.
And many of them also share the same requirements because there is a set of common questions, like \textit{``What are the side effects of drug X?''}.
But ignoring these shared requirements, many of the applications provide their own -- mostly proprietary -- knowledge bases.

A comparable situation on the data level of another domain was recently solved.
The Wikidata project ``centralizes access to and management of structured data'' \footnote{\url{https://www.wikidata.org/wiki/Wikidata:Main_Page}}.
It solves the problem that many different Wikipedia provided common data about the same entities, e.g. the population of countries.
Now Wikidata centrally provide this data through a common interface and the different Wikipedia refer to this source.
In consequence there exist only one place where the data has to be updated and maintained.


% \begin{itemize}
% \item verschiedene anwendungen im  bereich arzneimittelmngtm
% \item überschneidende funktionalität
% \item 
% \item keine endanwendungen auf lodd basierend bekannt
% \end{itemize}

\section{Problem}
\label{sec:problem}

In section \ref{sec:subject} the diversity of applications that implement drug related functionalities was mentioned.
This goes along with the fact that multiple knowledge bases are used which actually should provide the same data. 
This leads to several problems.

The first problem -- insufficient integration -- follows from the different scopes and the different expressivity of the knowledge bases.
This means there are many knowledge bases for special domains like side effects or alternative medicine.
But if one knowledge base wants to use the data of another special domain this is usually not possible, for instance because of an insufficient semantic integration.
This means that term $x$ in knowledge base $X$ does not implicitely refer to the same object as term $x$ in knowledge base $Y$.
Therefore the knowledge bases have to provide the data on their own.

This is the starting point of the next problem: redundancy.
Many drug related knowledge bases include a set of common facts that are essential for this domain, like the code of the \textit{Anatomical Therapeutic Chemical Classification System} (ATC).
So this is a common \textit{``Don't repeat yourself''} problem.
A side effect of this problem is the higher error rate that a given statement of a knowledge base is wrong.

So the third problem is the correctness of the knowledge.
Proprietary knowledge bases generally have to deal with a strong limited number of people validating the data.
In contrast open projects like Wikipedia are considered as ``self-healing'' information systems because of the high number of volunteers that report and correct data errors.
To transfer this phenomenon to medical data it is unimaginable that everyone could edit such important information, but reporting errors would be a strong impact nonetheless.
%While the number of persons that validate the data of a proprietary knowledge source is limited, projects like Wikipedia has shown the power of an open review process in the last years.

Section \ref{sec:subject} presented two third-party applications using open data as their knowledge sources.
A common problem using these data sources is the extensive schema knowledge that is required to use the data efficiently. 
Besides the information about the usage of the given interface the developers are often forced to understand the structure of the data to perform their respective queries.
This can be a deal-breaker for developers and therefore for innovative applications.
% \begin{itemize}
% \item unterschiedliche implementierungen
% \item unters. quellen
% \item ausgesagt werden soll aber das selbe
% \item 
% \item umfangreiches schemawissen nötig um anfragen an lod stellen zu können
% \end{itemize}

\section{Motivation}
\label{sec:motivation}

If the given problems of section \ref{sec:problem} could be solved this would lead to a better data quality in the first place.
Data quality contains in this context the expressivity, completeness and correctness of the knowledge.
Medical applications would benefit from this improvement by getting a proper knowledge foundation.
Following from that this would lead to unified, more well-grounded decisions for drug related questions.
This would hopefully avoid some of the prescription errors that are made during the healing process of a patient.

Solving the problem that developers often have to know the schema of a certain knowledge base the consequence would be a reduced starting barrier for application developers.
Instead of knowing about the whole schema of a knowledge base, only the knowledge about using the given interface would be required.
Probably this would lead to many interesting applications on top of the given data.

% \begin{itemize}
% \item durch einheitliche schnittstelle soll datenqualität vereinheitlicht werden
% \item einstiegsbarriere für neue anwendunge senken (cpoe)
% \item 
% \item vereinfachen der möglichkeit wissen aus lodd zu nutzen
% \item (praxiseinsatz von lodd)
% \end{itemize}

\section{Objectives}
\label{sec:objectives}

Emerged from the motivation, this thesis shall proof that Linked Open Data is an appropriate knowledge source for drug management tools.
For this purpose a web-based \textit{Application Programming Interface} (API) will be provided.
This API will support the process of answering drug related questions according to the WHO drug management life cycle.
This is achived by offering several API endpoints -- e.g. for drug-drug interactions -- that gather information from several semantic knowledge bases and return the merged results.
One design goal is the simplicity of the API to reduce the starting barrier of developing new applications based on the given interface.
More details about the implementation process are described in chapter \ref{cha:methods}.

This thesis shall also provide two use cases where the usage of the API will be demonstrated and evaluated.
The first use case is the integration in Dispedia -- an information system in the complex field of rare diseases \cite{elze2013dispedia}.
The main purpose here is the integration of the drug-drug interaction endpoint.
The second use case is a personal drug management portal built around this API to support privat persons managing their prescribed drugs.
This use case shall demonstrate the other available endpoints of the provided interface.
Chapter \ref{cha:evaluation} will offer more information about this evaluation process.

% \begin{itemize}
% \item bereit\-stellen einer API zur unter\-stützung des drug manage\-ment cycles basierend auf lod(d)
% \item evaluation anhand verschiedener anwendungsfälle
% \begin{itemize}
% \item dispedia
% \item eigenes portal zur arzneimittel-verwaltung
% \end{itemize}
% \end{itemize}

%%% Local Variables: 
%%% mode: latex
%%% TeX-master: "../thesis"
%%% End: 
